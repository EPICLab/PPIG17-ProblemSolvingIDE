\documentclass{ppig}
\usepackage{epsfig, graphicx} % support for image encoding and manipulation
\usepackage{ucs} % support for using UTF-8 as input encoding in LaTeX
\usepackage[utf8x]{inputenc} % required for UTF-8 support with ucs.sty
\usepackage{tabularx, multirow, booktabs} % support for high-quality tables
\usepackage{csquotes} % support for block quotes (using displayquote command)

% The titlebox defines how much vertical space is given for
% the authors' list. If you need extra space to show all the
% authors, uncomment the line below and increase the value. Please
% do not make the titlebox smaller than the original size of 5cm.
%\setlength\titlebox{5cm}

\title{An Examination of IDE Design for Programming as Problem-Solving}

% List the authors like you would in a table.
% The \And command creates another author's column. Use it after the
% details of one author to separate them from the following author horizontally.
% The \AND command creates a new "row" of authors and it should be used
% when the authors don't fit on the same line. You may have to increase
% the titlebox so that the author's don't overlap with the abstract.
\author{Nicholas Nelson \\
  Electrical Engineering \&\\ Computer Science \\
  Oregon State University \\
  nelsonni@oregonstate.edu \\
  \And
  Anita Sarma \\
  Electrical Engineering \&\\ Computer Science \\
  Oregon State University \\
  Anita.Sarma@oregonstate.edu \\
  \And
  André van der Hoek \\
  Department of Informatics \\
  University of California, Irvine \\
  andre@ics.uci.edu
}
  
\date{\today}

% Packages and macros for editorial purposes. Not required for submission.
\usepackage{color}
\definecolor{darkgreen}{rgb}{0.0, 0.5, 0.0}
\definecolor{ballblue}{rgb}{0.13, 0.67, 0.8}
\definecolor{aoblue}{rgb}{0.0, 0.0, 1.0}
\newcommand{\bold}[1]{\textit{\textbf{\color{aoblue}#1}}} % macro for boldifications
\newcommand{\todo}[1]{\textit{\textbf{\color{red}TODO: #1}}} % macro for TODO items
\newcommand{\discuss}[1]{\textit{\textbf{\color{darkgreen}#1}}} % macro for in-line discussions/questions
\newcommand{\nameUI}{\textit{<Insert Name>} UI} % macro placeholder for the name of the UI
\usepackage{enumitem}

\begin{document}
\maketitle
\thispagestyle{empty}

\begin{abstract}

Programming is inherently a problem-solving exercise: A programmer has to cultivate an understanding of the situation, externalize and contextualize thoughts \& ideas, develop strategies on how to proceed with the task, enact changes according to the most appropriate strategy, and retrospectively learn from each problem.
Therefore, programming is clearly more than just code input, testing, and maintenance.
Current IDEs, however, largely focus on the "writing code" parts of programming.
In this paper, we revisit which activities and actions constitute programming, and highlight six challenges to supporting these activities.
We then briefly describe a new paradigm of interacting with the IDE that we are working on to more directly support each of the six activities. 
\end{abstract}

\section{Programming as Problem-Solving}

Programming is more than dealing with language syntax and semantics: it is inherently an exercise in problem-solving that extends beyond the act of editing code in an Integrated Development Environment (IDE).
We are not the first to observe this.
For instance, programming has been characterized as an iterative process of refining mental representations of computational problems and solutions and expressing those representations as code~\cite{loksa2016programming}.

As an example, Leslie Lamport once opened a lecture with the statement, ``no one just starts writing code and hopes it happens to implement a web browser''~\cite{lamport2015lecture}.
He was advocating for design and specification prior to coding, but this statement also highlights the insight that programming requires thinking in a problem-solving manner at higher levels of abstraction than code.
Further evidence for the fact that programming is more than just coding is promoted by studies that have highlighted that programming requires gathering information from multiple sources~\cite{sillito2008asking}, includes creating mental models of program structures~\cite{von1995comprehension}, and involves exploring and evaluating many alternatives~\cite{hartmann2008design}.

% Much is known already about programming being a problem-solving activity beyond editing code in the editor provided by the IDE.
% During software development, for instance, programmers are known to create all sorts of auxiliary non-code artifacts~\cite{cherubini2007whiteboard}.
% As a second example, programmers are known to organize these artifacts into structures that are relevant to the particular task(s) they are currently focused upon~\cite{baltes2016empirical}.
% As a final example, programmers are known to not pursue a single solution, but to generally approach a task by exploring (either mentally or externalized) multiple alternative solutions~\cite{madeyski2017experimentation}.
% By applying prior work on problem-solving from a cognitive psychology perspective~\cite{mayer1992thinking}, we can classify these actions, respectively, into \textit{representing relevant information}, \textit{contextualizing information}, and \textit{generating alternatives}.
% These actions can then be generalized to represent the problem-solving activities of \textit{externalizing thoughts \& ideas} and \textit{developing strategies}.

\input{table_pps} % Programming as Problem-Solving Matrix

We have surveyed the literature from the perspective of programming as problem-solving, leading to Table~\ref{pps_matrix} which summarizes problem-solving activities that developers employ during programming sessions.
Problem-solving in programming partitions into six categories (\textit{Activities} column), with specific actions  that represent in more detail how the high-level activities manifest themselves (\textit{Actions} column).
Clearly, not every task involves all of these problem-solving actions, and there is no linearity to the order in which they are employed.
Sometimes these actions may not even be observable when an action is done in the programmer's head.
At the same time, literature has documented that these activities do occur and play an important role in how programmers arrive at a solution to the programming problem at hand.

\section{Challenges}\label{challenges}

Within this context, it is surprising that current IDEs only support some of this entire spectrum of programming activities, primarily focusing on support for navigating and editing the codebase.
Numerous challenges arise in current IDEs supporting the full set of actions that comprise problem-solving in programming.
These challenges span the different categories of activities (see Table~\ref{pps_matrix}, \textit{Activities} A1-A6), and even those activities (A4: \textit{enacting change}) that have traditionally been supported by IDEs have gaps in support when examined as problem-solving activities.

Supporting programmers in problem-solving is a difficult problem when we recognize that we need to understand human aspects such as how people think when programming, what development processes are required, what skills are best suited for the current problem, what knowledge already exists in relation to that problem, and what support (or lack thereof) already exists from previous problem solving efforts.
Therefore, creating an IDE that is meant to support all the different aspects of programming, from the problem-solving perspective, requires a careful analysis of both the individual aspects and their combinations.
We believe that it is necessary to fundamentally rethink IDEs.

To guide such a rethinking, the following challenges characterize what kind of novel support is needed for programming as problem-solving:

\begin{enumerate}
	\item \textit{\textbf{How to support programmers in formulating problems and reflecting on solutions?}}
	Programmers do not just arrive at a solution.
	They need to first \textit{contextualize} the computational problem in terms of what they know and how they can progress towards a possible solution.
	After which, they may explore, articulate, and reflect on different alternatives they try-- these activities are interleaved, sometimes even happening at the same time (e.g., a developer may reflect on the solution while they articulate it), and typically encompass several quick iterations.
	Often there is no single correct solution, and the best solution requires mixing and matching elements from multiple alternative solutions.
    
    \item \textit{\textbf{How to support programmers in viewing the relevant context in a problem space?}}
    No program is an island and the solution contained within it must exist in the context of the rest of the codebase and its related artifacts.
    Programmers need to understand where a code snippet fits in a codebase, what it calls out to, and what calls into it~\cite{desouza2008empirical}, the desired behaviors of the code snippet (e.g., computational speed, usability, feature sets), organizational policies (e.g., licensing, process standards, code style), and historical development (e.g., has a solution previously been tried and rejected).
    Information that defines the context is not always directly available, and instead, must be cobbled from different types of artifacts that span the codebase, design/requirement artifacts, communications records (i.e. handwritten, emails, group fileserver), edit history for project artifacts, etc.
    A developer needs to know where these individual pieces of information reside and how to cherry-pick the items that actually pertain to the problem at hand.
    
 	\item \textit{\textbf{How to support the variety of information processing and workflow styles of programmers?}}
 	Programmers exhibit creativity in their problem-solving activities, and the diversity of approaches means that no two programmers are the same.
 	Male programmers prefer to use a \textit{heuristic} (or \textit{selective}) approach that involves striving for efficiency by following contextually salient cues, whereas female programmers process information \textit{comprehensively}, seeking a more complete understanding of the problem~\cite{grigoreanu2012end}.
 	Visuospatial reasoning serves as a ubiquitous basis for abstract knowledge and inference, and is a core component to rationalizing the world around us~\cite{tversky2005visuospatial}.
 	When problem-solving, programmers bring their visuospatial reasoning and information processing style to bare when evaluating and organizing artifacts, solutions, and ideas.
  
  \item \textit{\textbf{How to support programmer understanding of the contextual history that lead to a solution?}}
  Problem-solving in programming does not occur in a vacuum, and neither is it a one-off activity.
  Programmers rely upon past experiences with similar problems, knowledge gained from previous interactions with the same set of artifacts, and memories of prior mental models that allowed comprehension and problem-solving to succeed.
  When solving a programming problem, programmers need to orient around both their own and others' problem-solving spaces in order to transition from contemplating into actualizing (enacting) a solution.
  
  \item \textit{\textbf{How to enable collaboration between programmers when problem-solving not just code, but all related artifacts?}}
  Organizing oneself in relation to a problem creates a cognitive load~\cite{sweller1988cognitive}, and this load only increases when one adds collaborators.
  When information is spread across multiple information sources (e.g., code, sketches, written notes), programmers quickly exceed the $7\pm2$ capacity of their short-term memory~\cite{lisman1995storage}.
  Therefore, programmers must conceptualize artifacts and information into abstractions that allow for both the pursuit of a solution and sharing in order to enable multiple problem-solvers to operate cohesively.
  
  \item \textit{\textbf{How to enable the use of all of this information and context to support the act of coding?}}
  Solutions to programming problems must eventually be represented in code.
  Therefore, programmers must convert their conceptual solutions into actual lines of code.
  This process is non-linear (i.e., coding sessions start and stop), concurrent with all other problem-solving activities, and loosely organized (e.g., solutions are partially implemented, abandoned, and recovered).
  Therefore, programmers must cope with high-complexity coding sessions in the pursuit of simple, elegant software solutions.
\end{enumerate}

\section{Toward A New IDE}

To address these challenges, we have begun a research effort that attempts to rethink the IDE from the ground up using the problem-solving perspective as a paradigm shift for evaluating our design rationale.
We use Code Bubbles~\cite{bragdon2010bubbles}, Code Canvas~\cite{deline2010canvas}, Debugger Canvas~\cite{deline2012debugger}, and Patchworks Code Editor~\cite{henley2014patchworks} as key inspirations for interface designs that eschew window-based interfaces and explore spatial interfaces allowing users to project meaning onto the layout of their development environment.
We also rely upon Lighthouse~\cite{dasilva2006lighthouse} as an inspiration for parallel development awareness in interface design.
Extending the zoomable canvas of Code Canvas and the bubble windows of Code Bubbles, plus the live information of Lighthouse, we explore these and other concepts from interface design to support the problem-solving activities that programmers encounter.

\begin{figure}[h!]
	\caption{Cards-based User Interface of a Problem-Solving IDE}
	\label{mockup}
	% trim={<left> <lower> <right> <upper>}
	\fbox{\includegraphics[trim={0.6cm 0.2cm 0.6cm 0.2cm},clip,width=\textwidth]{Mockup-10}}
	\vspace*{-1.5\baselineskip}
\end{figure}

We include the high-fidelity mockup in Figure~\ref{mockup} to illustrate our proposed user interface for a new IDE that is based upon the programming as problem-solving paradigm and targeted at the challenges described in Section~\ref{challenges}.

We propose a cards-based user interface, so that developers can take advantage of the neuro-cognitive process of perceptual organization~\cite{kimchi2003perceptual}.
Spatial perception allows developers to provide meaning to individual artifacts (i.e., cards) by organizing them in spatial patterns that contextualize the individual artifact, its relation to other artifacts, and the larger purpose.
As previously demonstrated, developers work with artifacts beyond code and therefore require cards that accommodate those non-code artifacts.
We propose that our cards would have individual types for different purposes; code editor cards for code artifacts (cards \texttt{C4}, \texttt{C5}), issue tracker cards for problem contextualization (cards \texttt{C1}, \texttt{C2}), image and PDF cards for design documents (card \texttt{C3}), and email viewer cards for communications (cards \texttt{C6-C8}).
This list is not exhaustive, and we expect that additional card types will be proposed and developed to accommodate additional problem-solving activities.

According to the Gestalt school of thought, our perception of the world is influenced by the way we group and segregate the visual stimuli presented to us~\cite{kimchi2003perceptual}.
The strength of grouping objects is inversely proportional to the distance between the elements~\cite{bergman2009peirce}; therefore, the closer the elements are to each other the stronger our perception is that these belong to the same group.
Therefore, we also propose the ability for users to group cards together into stacks of cards that can represent any logical unit of meaning that is necessary for working on the programming problem.

Since developers work together, in a variety of different capacities (see Table~\ref{pps_matrix}, \textit{Activity} A5), we also propose integrated methods for sharing individual cards (or stacks of cards) for either review or simultaneous editing.
This feature can be seen in cards ~\texttt{C4} and \texttt{C5}, which are shown as being shared between two different developers at the same time.
Edits from one developer will automatically updated the state of the shared cards in the IDEs of the other developer.
This sharing requires permissions and identity management which can be dynamically added and revoked as necessary for the programming problem.

The new IDE that we propose includes several new features in order to address some of the gaps that we have determined through examination of the programming as problem-solving paradigm are endemic to modern IDEs.
We hope to further develop these ideas through the development of a working prototype, and a subsequent study to validate the value of these features in addressing programming as problem-solving.

\bibliography{bibliography}
\bibliographystyle{apacite} 
\end{document}
